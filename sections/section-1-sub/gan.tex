\subsection{Генеративно-состязательные сети}
Генеративно-состязательные сети (англ. generative adversarial networks, GAN для краткости) -- это метод для генерирования данных с использованием средств глубокого обучения, таких, как, например, сверточные нейронные сети. Генерирование нового контента -- это задача обучения без учителя, которая включает в себя обнаружение и запоминание закономерностей или шаблонов во входных данных таким образом, чтобы впоследствии модель могла генерировать новые объекты. Сгенерированные данные должны быть достаточно правдоподобны, чтобы нельзя было определить, были ли они созданы при помощи нашей модели или же происходят из входного набора данных.

В сущности, генеративно-состязательные сети – это разумный способ свести задачу создания генеративной модели к задаче обучения с учителем, состоящей из двух подмоделей: генеративной модели, которая обучается созданию новых данных и дискриминативной модели, которая старается отличить «подлинные» образцы от сгенерированных. Две модели обучаются параллельно, постоянно пытаясь обыграть друг друга. Процесс обучения завершается тогда, когда дискриминативная модель ошибается приблизительно в половине случаев, что означает, что создаются достаточно правдоподобные образцы.

Впервые генеративно-состязательные сети были представлены в 2014 году исследователем Ian Goodfellow и его коллегами\cite{gan-introduction}, а некоторые дополнения, такие, как StyleGAN были разработаны совсем недавно -- в 2018 году\cite{gan-stylegan}. Этот вид сетей активно развивается, постоянно меняется и имеет уникальные возможности, в числе которых: создание фотореалистичных изображений заданных объектов (например, человеческих лиц), синтез голоса, поднятие разрешения фотографий\cite{gan-upscaling} или трансформацию одних изображений в другие.

В процессе обучения генеративная часть сети пытается минимизировать функцию (\ref{eq:gan-loss}), в то время как дискриминативная часть пытается её максимизировать. 
 \begin{equation}
 	\label{eq:gan-loss}
	E_x[log(D(x))] + E_z[log(1 - D(G(z)))]
 \end{equation}

Здесь:
\begin{itemize}
	\item $D(\cdot)$ -- вероятность того, что изображение является подлинным;
	\item $E$ -- мат. ожидание;
	\item $G(\cdot)$ -- вывод генератора;
	\item $x$ -- экземпляр из выборки подлинных данных;
	\item $z$ -- случайный шум.
\end{itemize}
%В последующих абзацах детали работы этих сетей будут описаны подробно.
