\subsection{Подготовка изображений с использованием Pix2Pix сети}
Чтобы значительно улучшить точность распознавания на финальном шаге в сверточной нейронной сети, сначала будем обрабатывать изображения captcha при помощи генеративно-состязательной Pix2Pix сети. Цель обработки изображений на этом шаге --удаление защитных признаков, таких, как искажения, шумы и линии, пересекающие текст.

Наша сеть GAN для подготовки изображений имеет обычную структуру, то есть состоит из генератора и дискриминатора. Цель обучения данной сети – научить ее очищать captcha от средств защиты и в целом стандартизировать входные картинки. Генератор работает с изображениями на пиксельном уровне и пытается их улучшить, в то время как дискриминатор пытается различить очищенные captcha от тех, которые изначально были созданы без защитных признаков.

Обучение сети проводится на парах изображений, созданных на предыдущем шаге: одно изображение -- captcha без защитных средств, второе изображение – та же самая картинка, но дополненная шумом, линиями и т. д. С течением процесса обучения, генератор будет создавать все более «чистые» картинки, а дискриминатор, в свою очередь, начинает лучше отличать очищенные изображения, от тех, которые были изначально без защитных средств. Процесс обучения останавливается тогда, когда дискриминатор станет ошибаться в достаточном проценте случаев, что означает, генератор достиг требуемого уровня очистки изображений.

После обучения сетей мы можем использовать генератор на новых данных и успешно очищать входные изображения от защитных средств.

При практическое реализации этих сетей будем использовать pix2pix фреймворк\cite{pix2pix-framework}, обеспечивающий удобную основу для создания необходимой нам сети. Изначально этот фреймворк направлен на преобразование изображений одного стиля в другой, однако он отлично подойдет и для решения наших задач. 
