\subsection{Обзор метода}
Вопрос автоматического решения текстовых captcha методами машинного обучения уже неоднократно поднимался исследователями в прошлом\cite{prev-research-2, prev-research-3, prev-research-4}. Тем не менее, большинство методов, представленных ранее, либо нацелены на решение captcha строго определенного типа, либо требуют большого набора данных для обучения (на создание которого как правило уходит большое количество времени и ресурсов). С эволюцией и усилением текстовых captcha некоторые предыдущие подходы теряют свою актуальность\cite{prev-research-1}.

В этой работе будут использоваться современные алгоритмы машинного обучения, в числе которых Pix2Pix-сети, SimGAN-сети и CNN-сети, для создания универсального метода решения captcha, не требующего большого вмешательства человека при обучении моделей.

Так, проблема нехватки данных для обучения может быть устранена путем автоматического создания большого количества картинок при помощи генератора с заведомо известным ответом для каждого созданного изображения captcha и последующего улучшения изображений при помощи генеративно-состязательной сети для повышения схожести с реальными данными. При использовании этого метода возможно обучить генеративную сеть созданию реалистичных текстовых captcha из искусственно сгенерированных изображений на небольшом наборе настоящих изображений, в данном случае, около 1 тысячи.

Процесс решения задачи автоматического решения captcha состоит из нескольких шагов:
\begin{enumerate}[1)]
	\item Создание небольшого набора действительных данных.
	
	На этом шаге производится сбор данных напрямую с целевых сайтов и производится решение каждой текстовой captcha. Для наших целей достаточно собрать таким образом не более 1000 объектов.
	\item Реализация генератора искусственных изображений captcha.
	
	Генератор изображений должен принимать на вход произвольное слово и набор защитных параметров, который должен быть применен к создаваемой captcha. На выход генератор выводит созданное изображение captcha.
	\item Улучшение искусственных изображений при помощи simGAN.
	
	С использованием данных из предыдущих шагов создается GAN сеть, которая должна генерировать максимально похожие на настоящие изображения, принимая на вход искусственно сгенерированные. Эта модель состоит из двух частей – генератора изображений и дискриминативной сети. Генератор изображений пытается создавать картинки, максимально похожие на реальные, а дискриминативная модель пытается разделять сгенерированные изображения от настоящих. Когда дискриминативная модель не может отличить большой процент сгенерированных изображений от настоящих, процесс обучения прекращается.
	\item Подготовка изображений к распознаванию при помощи Pix2Pix сети.
	
	На этом этапе происходит обработка изображений при помощи Pix2Pix сети, цель которой – стандартизировать входное изображение и очистить его от защитных средств, таких, как шум, пересекающиеся линии и символы. Обучение сети происходит с использованием данных, созданных на предыдущем этапе, сети демонстрируются изображения с включенными и отключенными защитными средствами, после чего она может работать с незнакомыми изображениями.
	\item Создание распознающей сверточной нейронной сети.
	
	На этом шаге создается сверточная нейронная сеть, которая обучается распознаванию символов с подготовленной на предыдущем шаге картинке. Обучение производится на подготовленном ранее большом наборе данных, который содержит как изображения, так и соответствующие им решения. Перед подачей изображения на вход нейронной сети производится его сегментация на символы.
\end{enumerate}

Далее эти этапы будут рассмотрены подробнее.