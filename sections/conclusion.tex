\section*{Заключение}
\addcontentsline{toc}{section}{\protect\numberline{}Заключение}
В рамках курсовой работы были получены следующие результаты:

\begin{enumerate}[1)]
	\item  Проанализированы существующие технологии в сфере глубокого обучения, а именно:
	\begin{itemize}
		\item сверточные нейронные сети;
		\item генеративно-состязательные нейронные сети (GAN);
		\item pix2pix сети как подвид GAN сетей;
		\item методы подбора оптимальных гиперпараметров.
	\end{itemize}
	\item Создано несколько датасетов изображений captcha для обучения нейронных сетей:
	\begin{itemize}
		\item датасет из 1000 изображений, созданный из данных реального сайта и решенный вручную;
		\item датасет из 200,000 изображений, созданный при помощи генератора на языке php с заранее известными ответами.
	\end{itemize}
	\item Рассмотрено практическое применение методов глубокого обучения на примере проблемы решения captcha. Решение состоит из нескольких этапов:
	\begin{itemize}
		\item подготовка наборов данных для обучения нейронных сетей;
		\item создание pix2pix генеративно-состязательной сети, отвечающей за очистку входных изображений от защитных признаков (шум, линии, дисторсия);
		\item сегментация изображения на символы с использованием метода контуров и библиотеки opencv;
		\item распознавание символов при помощи сверточной сети, похожей по структуре на LeNet 5;
		\item восстановление ответа из данных, полученных на предыдущих шагах.
	\end{itemize}
\end{enumerate}

В работе продемонстрирована эффективность и возможности современных методов глубокого обучения. Преимуществами такого подхода являются универсальность, возможность реализации при наличии очень небольшого объема не синтезированных данных и низкий уровень участия человека в процессе обучения.

Результаты работы показывают, что с развитием алгоритмов глубокого обучения падает эффективность текстовых captcha как средства защиты от ботов, процент ошибок хорошо обученных моделей приближается к значениям, получаемым при решении captcha человеком. Совершенствование сверточных и GAN сетей в области работы с изображениями актуализирует вопросы об изменении методов различения людей от ботов, смены текстовой captcha на более продвинутые методы защиты, например, видео captcha\cite{video-captcha} или игровая captcha\cite{prev-research-4}. Тем не менее, эволюция алгоритмов обучения с подкреплением подвергают сомнению эффективность и этих видов captcha.

Так возникает необходимость в создании совершенно новых типов captcha, возможно, комбинируя сразу несколько механизмов защиты. При этом требуется поддерживать простоту и удобство прохождения подобных тестов человеком.