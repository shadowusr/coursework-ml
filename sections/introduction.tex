\section*{Введение}
\addcontentsline{toc}{section}{\protect\numberline{}Введение}
В последние годы глубокие нейронные сети (включая рекуррентные) показали превосходство на различных соревнованиях и бенчмарках в сферах обработки естественного языка\cite{nlp-trends}, симуляции голоса \cite{gan-voice-synthesis}, компьютерного зрения\cite{cnn-imagenet} и генерирования реалистичных данных, обладающих заданными признаками. Это связано со многими факторами, в числе которых увеличение доступных вычислительных мощностей, появление новых, более эффективных методов и доступность больших объемов данных для обучения и тестирования нейронных сетей.

Темой данной работы является глубокое обучение, в частности, ге\-не\-ра\-тивно-состязательные нейронные сети, сверточные нейронные сети и Pix2Pix сети. В работе рассматривается практическое применение этих инструментов на примере задачи автоматического прохождения CAPTCHA\footnote{Completely Automated Public Turing test to tell Computers and Humans Apart, далее будет использоваться вариант в нижнем регистре.} тестов.

Актуальность работы обусловлена новизной вышеперечисленных методов, высокими результатами, которые они показывают, а также крайне большой популярностью текстовых captcha для разделения людей от компьютерных программ. Так, крупнейшие сервисы, включая Yandex, VK, Mail.ru, Google и другие, используют текстовые captcha для ограничения деятельности автоматизированных программ.

Цель работы -- детально рассмотреть новейшие методы глубокого обучения и их применимость к проблеме автоматического распознавания текстовых captcha.

Работа состоит из двух разделов: \hyperref[sec:section-1]{Обзор моделей глубокого обучения} и \hyperref[sec:section-2]{Распознавание текстовых captcha с использованием алгоритмов глубокого обучения}. В первом разделе будет представлена теоретическая основа и обзор используемых методов, а во втором – их практическое применение для решения проблемы распознавания изображений captcha.
